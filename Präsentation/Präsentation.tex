\documentclass[hyperref={pdfpagelabels=false}]{beamer}
\usepackage{lmodern}
\usepackage[german]{babel}
\usepackage[T1]{fontenc}
\usepackage[utf8]{inputenx}


\title{C++ SFINAE}   
\author{Kevin Gath} 
\date{\today} 

\begin{document}


\begin{frame}
\titlepage
\end{frame} 

%\begin{frame}
%\frametitle{Inhaltsverzeichnis}
%\tableofcontents
%\end{frame} 


\section{SFINAE} 
\begin{frame}
\frametitle{SFINAE} 
\begin{itemize}

\item SFINAE - \glqq{}Substitution failure is not an error\grqq
\begin{itemize}
\item Acronym wurde 2002 von David Vandevoorde eingeführt
\end{itemize}

\item Im Standard gibt es den Begriff nicht
\begin{itemize}
\item Der Standard beschreibt in § 14.8.2\footnote{\url{https://isocpp.org/std/the-standard}} den Sachverhalt
\item Ohne aber ein explizites Acronym zu verwenden
\end{itemize}

\end{itemize}

\end{frame}


%\subsection{Unterabschnitt Nr.1.1  }
%\begin{frame} 
%Denn ohne Titel fehlt ihnen was
%\end{frame}


\section{Worum geht es eigentlich?} 
%\subsection{Listen I}
\begin{frame}
\frametitle{Worum geht es eigentlich?}
\begin{itemize}
\item Situation

\begin{itemize}
\item Eine Menge von überladenenen Funktionen
\item Alle sind Kandidaten für einen potentiellen Funktions-Aufruf
\item Mindestens eine dieser Funktion ist ein Funktions-Template
\item Die Template-Argumente werden deduziert
\item Hierbei ergibt sich ein auf dem deduzierten Template-Typ beruhender Fehler in der Funktions-Schnittstelle
\begin{itemize}
\item Ein "Failure" beruhend auf der "Substitution"
\end{itemize}
\end{itemize}

\begin{itemize}
\item =>
\item Dies ist dann kein Compiler-Fehler
\begin{itemize}
\item Substitution Failure is not an Error
\end{itemize}

\item Sondern das Funktions-Template wird einfach aus der Menge der Kandidaten entfernt
\begin{itemize}
\item Der Compile-Vorgang läuft einfach weiter
\end{itemize}

\end{itemize} 
\end{itemize} 
\end{frame}















\end{document}
