\documentclass{beamer}
\usepackage{lmodern}
\usepackage[german]{babel}
\usepackage[T1]{fontenc}
\usepackage[utf8]{inputenx}
\usepackage{listings}
\usepackage{wrapfig}
\usepackage{hyperref}
\usepackage{color}
\usepackage{lscape}
\usepackage{courier}


\definecolor{javared}{rgb}{0.6,0,0} % for strings
\definecolor{javagreen}{rgb}{0.25,0.5,0.35} % comments
\definecolor{javapurple}{rgb}{0.5,0,0.35} % keywords
\definecolor{javadocblue}{rgb}{0.25,0.35,0.75} % javadoc
 
\lstset{language=C++,
keywordstyle=\color{javapurple}\bfseries,
stringstyle=\color{javared},
commentstyle=\color{javagreen},
morecomment=[s][\color{javadocblue}]{/**}{*/},
tabsize=2,
showspaces=false,
showstringspaces=false,
basicstyle=\scriptsize\ttfamily}





\title{C++ SFINAE}   
\author{Kevin Gath} 
\date{\today} 

\begin{document}


\begin{frame}
\titlepage
\end{frame} 

%\begin{frame}
%\frametitle{Inhaltsverzeichnis}
%\tableofcontents
%\end{frame} 


\section{SFINAE} 
\begin{frame}
\frametitle{SFINAE} 
\begin{itemize}

\item SFINAE - \glqq{}Substitution failure is not an error\grqq
\begin{itemize}
\item Acronym wurde 2002 von David Vandevoorde eingeführt
\end{itemize}

\item Im Standard gibt es den Begriff nicht
\begin{itemize}
\item Der Standard beschreibt in § 14.8.2\footnote{\url{https://isocpp.org/std/the-standard}} den Sachverhalt
\item Ohne aber ein explizites Acronym zu verwenden
\end{itemize}

\end{itemize}

\end{frame}


%\subsection{Unterabschnitt Nr.1.1  }
%\begin{frame} 
%Denn ohne Titel fehlt ihnen was
%\end{frame}


\section{Worum geht es eigentlich?} 
%\subsection{Listen I}
\begin{frame}
\frametitle{Worum geht es eigentlich?}
\begin{itemize}
\item Situation

\begin{itemize}
\item Eine Menge von überladenenen Funktionen
\item Alle sind Kandidaten für einen potentiellen Funktions-Aufruf
\item Mindestens eine dieser Funktion ist ein Funktions-Template
\item Die Template-Argumente werden deduziert\footnote{\url{http://en.cppreference.com/w/cpp/language/template_argument_deduction}}
\item Hierbei ergibt sich ein auf dem deduzierten Template-Typ beruhender Fehler in der Funktions-Schnittstelle
\begin{itemize}
\item Ein \glqq{}Failure\grqq{} beruhend auf der \glqq{}Substitution\grqq
\end{itemize}
\end{itemize}

\begin{itemize}
\item =>
\item Dies ist dann kein Compiler-Fehler
\begin{itemize}
\item Substitution Failure is not an Error
\end{itemize}

\item Sondern das Funktions-Template wird einfach aus der Menge der Kandidaten entfernt
\begin{itemize}
\item Der Compile-Vorgang läuft einfach weiter
\end{itemize}

\end{itemize} 
\end{itemize} 
\end{frame}


\section{Ende} 
%\subsection{Listen I}
\begin{frame}
\frametitle{Das war's}
\begin{itemize}
\item \glqq{} substitution failure is not an error\grqq
\item Fertig
\end{itemize}
\end{frame}


\section{Beispiele1} 
%\subsection{Listen I}
\begin{frame}
\frametitle{C++ SFINAE - Beispiel I}

\begin{itemize}
\item Also eine Menge von überladenen Funktionen
\begin{itemize}
\item Alle sind Kandidaten für einen potentiellen Funktions-Aufruf
\end{itemize}

\item Mindestens eine dieser Funktion ist ein Funktions-Template
\end{itemize}

\end{frame}

\begin{frame}[fragile]
\frametitle{C++ SFINAE - Beispiel I}
\begin{lstlisting}
void print(long l) 
{ 
   cout << "l: " << l << endl; 
}

template<class T> void print(T t)
{
   cout << "T: " << t << endl;
}

template<class T> void print(T* t)
{
   cout << "T*: " << *t << endl;
}
\end{lstlisting}


\begin{lstlisting}
int n = 42;
print(n);    // => T: 42
print(&n);   // => T*: 42
print(43L);  // => l: 43
\end{lstlisting}
\end{frame}


\section{Beispiel1Fortsetzung}
\begin{frame}
\frametitle{C++ SFINAE - Beispiel 1 - Fortsetzung}

\begin{itemize}
\item Die Template-Argumente werden deduziert
\item Hierbei ergibt sich ein auf dem deduzierten Template-Typ beruhender Fehler in der Funktions-Schnittstelle
\begin{itemize}
\item Ein \glqq{}Failure\grqq{} beruhend auf der \glqq{}Substitution\grqq{}
\end{itemize}

\item => Dies ist dann kein Compiler-Fehler
\begin{itemize}
\item Substitution Failure is not an Error
\item Das Funktions-Template wird einfach aus der Menge der Kandidaten entfernt
\item SFINAE  schlägt während der Funktions-Überladen Auflösung zu
\end{itemize}

\end{itemize}
\end{frame}

\begin{frame}[fragile]
\frametitle{C++ SFINAE - Beispiel I - Fortsetzung}
\begin{lstlisting}
void print(long l){ cout << "l: " << l << endl; }

template<class T> void print(T t){ cout << "T: " << t << endl; }

template<class T> void print(T* t){ cout << "T*: " << *t << endl; }


// Zusaetzlich

template<class T> void print(typename T::type t)
{
   cout << "T::type: " << t << endl;
}

// => Kein Problem
\end{lstlisting}


\begin{lstlisting}
int n = 42;
print(n);    // => T: 42
print(&n);   // => T*: 42
print(43L);  // => l: 43
\end{lstlisting}
\end{frame}

\begin{frame}[fragile]
\frametitle{C++ SFINAE - Beispiel 1 - Fortsetzung}

\begin{itemize}
\item SFINAE wurde genau dafür eingeführt
\item Bestehender Code sollte nicht ungültig werden, wenn zusätzlich (z.B. durch einen erweiterten Header) ein Funktions-Template in die Menge der potentiellen Aufruf-Kandidaten hinzukommt, das nach der Typ-Deduktion nicht \glqq{}okay\grqq{} ist.
\end{itemize}

\begin{lstlisting}
// Zusaetzlich

template<class T> void print(typename T::type t)
{
   cout << "T::type: " << t << endl;
}

// => Kein Problem
\end{lstlisting}

\end{frame}


\begin{frame}[fragile]
\frametitle{C++ SFINAE}
Ganz nebenbei:

\begin{itemize}
\item Wie spricht man unser neues Funktions-Template an?
\end{itemize}

\begin{lstlisting}
struct A
{
   typedef bool type;
};

int main()
{
   cout << boolalpha;

   int n = 42;
   print(n);         // => T: 42
   print(&n);        // => T*: 42
   print(43L);       // => long: 43
   print(false);     // => T: false
   print<A>(true);   // => T::type: true
}
\end{lstlisting}

\end{frame}


\end{document}
